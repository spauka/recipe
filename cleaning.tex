\section{Cleaning}

\subsubsection{Aim}
Remove any gallium, dust and surface contamination from the chip

\subsubsection{Needs}
\begin{itemize} [noitemsep]
\item AZ6612
\item NMP
\item Acetone
\item Isopropanol (IPA)
\item clean tweezers
\item wipes
\item beakers for IPA, NMP
\end{itemize}

\subsubsection{Steps}
\begin{enumerate}
\item Warm up a hot plate to \SI{200}{\celsius}
\item Place a few drops of AZ6612 onto the tip of a glass slide. Place chips face down on the AZ6612 and bake at \SI{110}{\celsius}
      for \SI{2}{\minute}.
\item With a clean room q-tip, brush the back of the chip to remove gallium. If necessary, the chips can be reheated to remelt the gallium.
\item Once clean, place the glass slide in a waste beaker of NMP to detach the chips from the slide.
\item Fill another beaker with NMP with enough liquid to make it stand in the sonicator.
\item Put GaAs face up in clean NMP. Warm for 10 min at \SI{80}{\celsius} (Hot plate should be at correct temperature already).
\item Add some water to the sonicator if needed.
\item Sonicate for 5 min (press button in middle of sonicator to turn on) in NMP.
\item Move chip to a beaker of acetone and sonicate in acetone for 5 min.
\item Move chip to a beaker of IPA and sonicate in IPA for 5 min.
\item Blow dry with nitrogen on wipe while holding the chip with tweezers.
\item Bake on center of hot plate for 5 min at \SI{200}{\celsius}.
\item Put chip back in box.
\item Pour solvents in waste solvent container, and rinse beakers with IPA and blow dry.
\end{enumerate}

\subsubsection{Misc}
\begin{itemize}
\item The variable temperature hot plate takes a long time to warm up. Turn it on as early as possible.
\item Put filter papers between box and beakers, in case the chip drops
\item \textbf{Do not place acetone or IPA on the hot plates}
\item Clean tweezers between steps
\end{itemize}

\newpage

