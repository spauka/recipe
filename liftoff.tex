\section{Metal lift-off}

\textbf{Aim}:
Remove the remaining resist and metal after deposition

\subsubsection{Needs}
\begin{itemize}[noitemsep]
\item NMP and corresponding beaker
\item IPA and corresponding beaker
\item Acetone and corresponding beaker
\item water and corresponding beaker
\item sonicator
\end{itemize}

\subsubsection{Steps}
\begin{enumerate}
\item Warm NMP for at least 10 min to \SI{70}{\celsius}.
\item Ensure that there is sufficient water in the sonicator (should come well over stage).
\item Place chips in warm NMP for 1 hour.
\item Sonicate at minimum power for \SI{5}{\second}. (Can use medium power, \SI{10}{\second} for ohmics).
\item While in NMP, check to see if liftoff looks successful. Do not allow the chip to dry.
\item If unsuccessful either repeat above steps, or spray across surface of chip with acetone. Repeat
as many times as necessary.
\item Put chip in IPA beaker, spray with bottle.
\item Blow dry well.
\end{enumerate}

\noindent \textbf{Misc.}:
\begin{itemize}
\item If liftoff was unsuccessful, you can attempt the above steps again. However once dry, metal becomes
much more difficult to lift off.
\item Spraying vigorously with acetone can help liftoff along.
\item If gates are broken, check the edges under SEM. Bright white edges suggests the metal was ripped
off during liftoff and is an indicator of insufficient resist depth.
\end{itemize}
\newpage

