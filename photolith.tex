\section{Photolithography}

\subsubsection{Aim}
Weaken the non-exposed resist

\noindent \textbf{Need}:
\begin{itemize}[noitemsep]
\item chip
\item tweezers
\item clean mask (upper side: silver, bottom side: chrome)
\end{itemize}

\subsubsection{Steps - Quintel Q6000}
\begin{enumerate}
\item At the bottom left part, check that the device is in:
\begin{itemize}[nolistsep, noitemsep]
\item Constant power mode ($\neq$ constant intensity)
\item Watts mode ($\neq$ \si{\milli\watt\per\square\centi\meter})
\end{itemize}
\item Turn power on lamp on (bottom right switch)
\item Press and hold the start button (around 2 sec) until the sound `wiiiii' vanishes.
Device is powered up when levels on left/bottom gauges have moved from zero.
\item Wait 10 min for the lamp to warm up
\item Switch on the power to the aligner
\item At the bottom left, set up modes to:
\begin{itemize}[nolistsep, noitemsep]
  \item Constant Intensity
  \item \si{\milli\watt\per\square\centi\meter}
\end{itemize}
\item Check that the aligner is indicating pressure mode
\item Hit \fbox{exposure-test} switch to make place and check the exposure properties
\item Check the exposure time and intensity:
\begin{itemize}[nolistsep, noitemsep]
  \item Exposure time at 1.6 sec
  \item Check the wavelength is \SI{365}{\nano\meter}
  \item Hit \fbox{shutter} on/off to check the power which should be 10 mW (to be read on one of the bottom left gauges)
  \item Hit \fbox{pulse} to check and save the exposure time
  \item After the pulse, the lamp will move out of the way and the microscope into position
\end{itemize}
\item Hit the mask \fbox{load} button to move the microscope out of the way
\item Put the chip on the right stage
\item Hit \fbox{load} to load the chip at the place where the mask will be
\item Manually position the mask above the chip without touching the chip and without lying the mask on the vacuum line
\item Unload the chip (hitting \fbox{clear})
\item Put the mask on the vacuum line
\item Hit \fbox{vac} to clamp the mask with vacuum
\item Load the chip
\item Swing the microscope back (mask \fbox{load})
\item Wait for the light corresponding to "contact" to be green (chip can be moved) and not red (chip must not be moved).
(Note: if the light does not illuminate, swing the microscope out and back)
\item Turn the \fbox{lamp~saver} off
\item Adjust the chip position under the mask using joysticks (take care of the maximum displacement range of joysticks)
\item Press \fbox{contact} to make the mask in contact with the chip
\item Press \fbox{exposure}. The chip will automatically swing out
\item Unload the chip, put it back in its box
\item Press \fbox{load} to put the microscope on the right
\item Press \fbox{vac} to free the mask
\item Put the mask in its box, ensure it is firmly closed
\item Turn off powers on top, then bottom of the device
\item Switch modes to:
\begin{itemize}[nolistsep, noitemsep]
\item Constant Power
\item Watts
\end{itemize}
\end{enumerate}
Note: If the 'separate' light does not illuminate, move the microscope back and forward (hit the mask \fbox{load} button twice).
\newpage
\subsubsection{Steps - Karl Suss MA6}
\begin{enumerate}
\item Check the three missile switches on the back of the mask aligner are in the \fbox{ON} position.
\item Start the flow of compressed air to the machine.
\item Start the lamp. On the lamp controller:
\begin{enumerate}
  \item Press the \fbox{ON} button. Wait until the screen shows "READY".
  \item Press the \fbox{CP} button. Wait until the screen shows "IGNITION".
  \item Press the \fbox{Start} button.
  \item The screen will show "Lamp Cold" and an error light will flash for ~3 minutes.
  \item After the lamp is warm, power should read \SI{299}{W}.
\end{enumerate}
\item Toggle the missile switch on the front of the mask aligner to turn it on. Press the \fbox{Load}
      button to start.
\item Turn off the \fbox{BSA Microscope} feature on the mask aligner.
\item Load the mask:
\begin{enumerate}
  \item Check that the mask plate is the correct size. We use the 4 inch wafer plate.
  \item Press the \fbox{Change Mask} button.
  \item Place the mask on the mask plate, chrome side facing upwards. Press the \fbox{Enter} button to clamp the mask.
  \item Check the mask is secure. Slot the mask plate into the aligner.
  \item Press the \fbox{Change Mask} button to secure the plate and complete the procedure. \textbf{Warning: Do not press
        \fbox{Enter} once the plate is in the aligner, as the mask will drop.}
\end{enumerate}
\item Set up the exposure:
\begin{itemize} [noitemsep, nolistsep]
  \item \textbf{To start a new program}
  \begin{enumerate} [noitemsep, nolistsep]
    \item Push \fbox{Select Program}.
    \item Use the \fbox{$\uparrow$} and \fbox{$\downarrow$} keys to select a contact mode. We generally use \textbf{Hard} contact mode.
    \item Push \fbox{Select Program} and move on to the edit parameters step.
  \end{enumerate}
  \item \textbf{To load a previously stored program}
  \begin{enumerate} [noitemsep, nolistsep]
    \item Push \fbox{Edit Program} to move into the select program mode.
    \item Using the \fbox{$\leftarrow$} and \fbox{$\rightarrow$} keys, select the "Load Program [x]" option.
    \item Using the \fbox{$\uparrow$} and \fbox{$\downarrow$} keys, select the correct program number.
    \item Push \fbox{Edit Program} to load the program.
  \end{enumerate}
  \item \textbf{To edit parameters}
  \begin{enumerate} [noitemsep, nolistsep]
    \item Push \fbox{Edit Parameter} to move into the select parameters mode.
    \item Using the \fbox{$\leftarrow$} and \fbox{$\rightarrow$} keys, select the parameter you wish to change.
    \item Using the \fbox{$\uparrow$} and \fbox{$\downarrow$} keys, change the parameter. The following parameters
          have worked well.
    \begin{itemize} [noitemsep, nolistsep]
      \item Exposure Time: \SI{1.7}{\second}
      \item Alignment Gap: \SI{60}{\micro\meter}
      \item HC Wait Time: \SI{10}{\second}
      \item Contact Mode: Hard
    \end{itemize}
    \item Once finished, press \fbox{Edit Parameter} to exit the parameter selection mode.
  \end{enumerate}
  \item \textbf{To save the program}
  \begin{enumerate} [noitemsep, nolistsep]
    \item Push \fbox{Edit Program}
    \item Using the \fbox{$\leftarrow$} and \fbox{$\rightarrow$} keys, select the "Save Program [x]" option.
    \item Using the \fbox{$\uparrow$} and \fbox{$\downarrow$} keys, select the correct program number.
    \item Push \fbox{Edit Program} to save the program.
  \end{enumerate}
\end{itemize}
\item Load the chip:
\begin{enumerate} [noitemsep, nolistsep]
  \item Check that the wafer stage is the correct one for your wafer. We use the 2 inch wafer stage. Ensure that
        the outer two vacuum mounts are off.
  \item Press \fbox{Load} to start the chip load procedure.
  \item Pull the wafer plate from the mask aligner. Place the chip onto the stage. If the chip is less than
        $6 \times 6 \si{\milli\meter}$, cut a square of blue dicing tape and place the chip onto the tape before loading it.
  \item Push the wafer stage back into the aligner, and press \fbox{Enter}. The wafer stage will come into contact with the mask.
        Wait until the screen shows "Align substrates".
\end{enumerate}
\item Align the substrate:
\begin{enumerate} [noitemsep, nolistsep]
  \item The microscope in use can be changed using the switch under the eyepiece. Focus is adjusted using either the coarse knob at the back of the scope 
        assembly or the fine knob at the front of the scope.
  \item Align the substrate using the two large micrometers on either side of the mask aligner (Left Y, Right X), and the small micrometer at the front right ($\theta$).
  \item The alignment gap can be changed in small steps using the \fbox{$\uparrow$ Sep} and \fbox{$\downarrow$ Sep} buttons.
  \item Once aligned, press \fbox{Alignment Check} to ensure the chip doesn't shift when it comes into contact with the mask. Press again to move back out of alignment and adjust.
\end{enumerate}
\item Once you are happy, press \fbox{Exposure}. Once complete, pull out the wafer stage and unload the chip.
\item Repeat from the Load Chip section for all chips you wish to expose.
\item To unload the mask:
\begin{enumerate} [noitemsep, nolistsep]
  \item Press the \fbox{Load Mask} button.
  \item Pull out the mask plate.
  \item Press \fbox{Enter} to remove the mask vacuum and remove the mask.
\end{enumerate}
\item To turn off the mask aligner:
\begin{enumerate} [noitemsep, nolistsep]
  \item Toggle the mask aligner switch to Off
  \item Press the \fbox{OFF} button on the lamp power supply.
  \item Wait 10 minutes for the lamp to cool.
  \item Turn of the compressed air to the aligner.
\end{enumerate}
\end{enumerate}
\textbf{Notes:}
\begin{itemize} [noitemsep, nolistsep]
  \item The lamp must be kept off for at least 30 minutes after it was last used. If another user wishes to use the aligner
        or you wish to use the aligner in a short period of time, it is best to leave the lamp on.
  \item To unload a chip after it has been loaded but without exposing it, press the \fbox{Unload} button.
  \item The main mask aligner power does not need to be turned off after use.
\end{itemize}
\newpage
\subsubsection{Misc}
\begin{itemize}
\item Exposed resist is removed (negative resist)
\item To clean the mask, spray IPA on both sides, gently wipe with clean room wipes. Blow dry with nitrogen.
\item Turn the lamp on 5 mins before you need to align, it needs time to start up.
\end{itemize}
\newpage


