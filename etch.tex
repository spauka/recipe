\section{Mesa etching}

\textbf{Aim}:
Remove the 2DEG layer which is not under the mesa pattern. The thickness to remove is around \SI{210}{\nano\meter} for Sandia, and \SI{110}{\nano\meter} for Gossard
and \SI{130}{\nano\meter} for Manfra.

\subsubsection{Needs}
\begin{itemize}[noitemsep]
\item Mesa Etch, DI Water beakers
\item measuring cylinders for \ce{H2O2}, \ce{H2SO4}, \ce{H2O}
\item one big beaker
\item 2 pipettes
\item timer
\end{itemize}

\subsubsection{Steps}
\begin{enumerate}
\item Open the water tap and let it flow
\item Be sure your gloves are fitted properly (you might want to double glove)
\item Prepare the acid etch solution. Remember \underline{always acid in water}. The recipe is:
  \begin{itemize}
  \item 240ml of \ce{H2O}
  \item 8ml of \ce{H2O2}
  \item 1ml of \ce{H2SO4}
  \end{itemize}
\item Rinse used beakers and measuring cylinders thoroughly. Acids and \ce{H2O2} go down the drain, not in the solvents.
\item Mix acid solution with tweezers
\item Put some of the solution in the 'mesa etch beaker'
\item Put the chip in the solution for 50 sec (the etch rate should be around 2nm/sec)
\item Put the chip in water, rinse thoroughly for 30 sec.
\item Blow dry
\end{enumerate}
\subsection{Measure the etch depth}
\begin{itemize}
\item Use the Dektak to measure the depth of the etch
\item Measure at the same points than when the height of the resist was measured. The height of the resist is not uniform around the chip so it
is important to use the same locations when measuring.
\item Use measurements before etch to compute the etching rate (depth after etch (\AA) - depth before etch (\AA))/ etching time (sec) = etching rate ((\AA)/sec)
\item Deduce the etching time needed to etch at a total depth of 210 nm (etching more will reveal impurities)
\end{itemize}
\newpage

