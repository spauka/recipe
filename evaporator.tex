\section{Metal Deposition}
\subsubsection{Aim}
Deposit metal on device to cover patterns not covered with resist.

\subsubsection{Needs}
\begin{itemize}[noitemsep]
  \item materials and corresponding boats
  \item chip holder
  \item safe work procedure of the evaporator
  \item round shaped wipes
\end{itemize}

\newpage

\subsection{Recipes}

\subsubsection{Ohmics}
\begin{itemize}
\item $2 \times 3\text{---}4$ pieces of \SI{2.5}{\milli\meter} of Ni (very reactive)\\
  (Note: Use two boats in case one fails)
\item 3 pieces of \SI{2.5}{\milli\meter} of Ge (does not degaz, very sensitive)
\item 8 pieces of \SI{2.5}{\milli\meter} of Au (does not degaz, slow heat transfer)
\end{itemize}

\textbf{Recipe: }
\begin{enumerate}[label=\protect\nth{\value*} layer: ,noitemsep,leftmargin=10em]
\item \SI{50}{\angstrom} of Ni at \SI{0.5}{\angstrom\per\second} 
\item \SI{350}{\angstrom} of Ge at \SI{1}{\angstrom\per\second} 
\item \SI{720}{\angstrom} of Au at \SI{2.5}{\angstrom\per\second} 
\item \SI{180}{\angstrom} of Ni at \SI{1}{\angstrom\per\second} 
\item \SI{500}{\angstrom} of Au at \SI{1.5}{\angstrom\per\second} 
\end{enumerate}

\subsubsection{Fine Gates}
\begin{itemize}
\item 4 pieces of \SI{2.5}{\milli\meter} of Ti
\item 3 pieces of \SI{2.5}{\milli\meter} of Au
\end{itemize}

\textbf{Recipe: }
\begin{enumerate}[label=\protect\nth{\value*} layer: ,noitemsep,leftmargin=10em]
\item \SI{80}{\angstrom} of Ti at \SI{1}{\angstrom\per\second}
\item \SI{120}{\angstrom} of Au at 3 to \SI{4}{\angstrom\per\second}
\end{enumerate}


\subsubsection{Contacts deposition}
\begin{itemize}
\item 4 pieces of \SI{2.5}{\milli\meter} of Ti
\item 7 pieces of \SI{2.5}{\milli\meter} of Au
\end{itemize}

\textbf{Recipe: }
\begin{enumerate}[label=\protect\nth{\value*} layer: ,noitemsep,leftmargin=10em]
\item \SI{100}{\angstrom} of Ti at \SI{1}{\angstrom\per\second}
\item \SI{2000}{\angstrom} of Au at 3 to \SI{4}{\angstrom\per\second}
\end{enumerate}

\newpage

\subsection{Thermal Lesker}
\subsubsection{Preparation}
\begin{enumerate}
\item Check that the pump is off
\item Vent for 3 min
\item Lift the bell with the remote and remove shield
\item Place boat on poles using the tools. Boats should fit between the circular washers. Tighten gently, ensuring that
boats do not twist. Boats should also be manipulated on clean room wipes when not in the evaporator.
\item Place cut metals into the appropriate boats. Remember to double glove and clean tweezers between each metal
to prevent cross-contamination. Sources are found in the "Evaporator Source Metals" drawer and come with corresponding cutters.
\item Using the lever on the right hand side, place the sample holder above the right boat.
\item Check the oscillator capacity (press \boxed{xtal} on the infinicon monitor. will return to home screen after a few seconds). Replace if
oscillator capacity is greater than 25\%.
\item Insert the sample holder into holder. Ensure that the holder locks firmly into place. The chip can either be clamped on a corner of the chip,
      or alternatively using a drop of PMMA on a circular glass slide.
\item Place boat separators between each boat. Ensure they do not touch the evaporator poles.
\item Before closing check:
\begin{itemize}[nolistsep,noitemsep]
  \item chip position
  \item boat + material
  \item oscillator
  \item boat separators don't touch boats holders (short circuits)
\end{itemize}
\item Replace shield
\item Clean the seal and contact area below bell with IPA
\item Bring the bell down using the remote. Do not bring it below the indicator.
\item Turn on the mechanical pump (big toggle switch)
\item When the pressure is $\leq$ \SI{0.2}{\milli\bar}, hit \boxed{start} to turn the turbo on\\
\item \underline{when pressure $\leq$ 10$^{-2}$ mbar} (wait 10 minutes after turbo start), the pressure can be read by turning the filament on and then off as soon as the pressure is read.\\
\item Wait at least 1 hour for pumpdown.
\end{enumerate}

\subsubsection{Deposition}
\begin{enumerate}[resume]
\item Check the pressure in the evaporator using the filament gauge. Record this chip in the evaporator logbook.
\item Position chips above source using the rotary switch on the right of the lesker.
\item Select the boat using the power selector at the bottom of the lesker (should make *click* when it locks in to place)
\item Select the correct film on the oscillator:
\begin{enumerate}[nolistsep,noitemsep]
  \item Hit \boxed{pgm}
  \item Hit \boxed{C} to go up one level to the program select position (above number on left)
  \item Hit \boxed{X}, where X is the wanted program number
  \item Hit enter \boxed{E}
  \item Hit \boxed{pgm} to exit
  \item Hit \boxed{zero} to zero the thickness
\end{enumerate}
\item Ensure that the power supply is dialled down
\item Turn the source power on
\item Turn on the filament to read the pressure in real time
\item Ramp up the power at a rate of \SI{10}{\percent\per\second} until the thickness monitor begins to read \SI{0.1}{\angstrom\per\second}.
\item Wait at this point for the material to degaz for a few seconds. The pressure will rise suddenly.
\item Open the source shutter, hit \boxed{zero} on the thickness monitor at the same time. Adjust deposition rate to the correct value. 
\item Note the pressure Pevap of evaporation
\item If needed, adjust the power to keep it constant. !Different metals have different response times! 
\item Close the shutter when the wanted thickness is reached
\item Bring the power to zero (can be done quickly)
\item Turn the power off
\item Repeat for each metal. If possible, wait for vacuum to recover to $< \SI{2d-6}{\milli\bar}$ between each evaporation.
\end{enumerate}

\subsubsection{Finishing}
\begin{enumerate}[resume]
\item Wait 5 min for the sample and sources to cool down.
\item Turn off the filament gauge.
\item Turn off the turbo and the pump.
\item Wait 5 min.
\item Vent for 3 min.
\item Remove and clean everything.
\item Put boats in contaminated waste (note that gold boats should be reused).
\item Update logbook.
\end{enumerate}

\subsubsection{Misc}
\begin{itemize}
\item Always check before each step:
\begin{itemize}[noitemsep,nolistsep]
  \item power position
  \item program number
  \item sample position
\end{itemize}
\item never dial power above $50\%$.
\item deposition rate: never more than \SI{5}{\angstrom\per\second}
\item pump down for as long as possible. 1 hour is the minimum.
\item the pressure \SI{2.10d-6}{\torr} is reached in about 3 hours
\item 4 sticks of Ti can make a layer with a thickness $\approx$ \SI{80}{\angstrom}
\item 3 sticks of gold (2.5\,mm long) can make a layer with a thickness $\approx$ \SI{500}{\angstrom}
\item When reusing boats, be aware that boats become brittle after use.
\end{itemize}

\newpage

\subsection{E-Beam Evaporator}
