\section{Resist developing}

\textbf{Aim}:
Remove the resist weakened by photolithography

\subsubsection{Needs}
\begin{itemize}[noitemsep]
\item exposed chip
\item resist developer MIF326 for AX6612, MIF826 for nLOF2020 or MIBK:IPA 1:3 for PMMA A3
\item developer beaker
\item water/IPA beaker
\end{itemize}

\subsubsection{Steps}
\begin{enumerate}
\item Post-Bake nLOF2020 at \SI{110}{\celsius} for \SI{60}{\second}.
\item Pour developer in the appropriate beaker
\item Prepare a beaker of DI water (for TMAH based developers) or IPA (for MIBK) for rinsing
\item Agitate the chip in the developer for the prescribed time.
\item Agitate the chip in water/IPA for the prescribed time to rinse it.
\item Blow dry for 10 seconds
\item Check under microscope
\item If islands of resist remains, repeat the operation for a shorter time.
\item Dispose of developer and rinse beakers with water and IPA
\item \underline{MIF Developer should be disposed in the MIBK waste container}
\end{enumerate}

The times for developing are as follows:
\begin{itemize}
  \item \textbf{nLOF2020: } 50 second develop, 30 second rinse
  \item \textbf{AZ6612: } 55 second develop, 30 second rinse
  \item \textbf{PMMA A3: } 40 seconds develop, 20 second rinse
\end{itemize}

\subsubsection{Misc}
\begin{itemize}
  \item The aim is to get sharp well defined edges, of the same size as on the mask.
  \item If features are smaller than on the mask, you can either increase exposure time or increase development time.
  \item If features are too large, either decrease exposure time or development time.
  \item The size of the undercut on LOR is defined by the amount of time it spends in developer (and bake temperature).
  It is not affected by exposure time.
\end{itemize}
\newpage

