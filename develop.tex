\section{Resist developing}

\textbf{Aim}:
Remove the resist weakened by photolithography

\subsubsection{Needs}
\begin{itemize}[noitemsep]
\item exposed chip
\item resist developer MIF300
\item MIF300 beaker
\item water beaker
\end{itemize}

\subsubsection{Steps}
\begin{enumerate}
\item Pour developer in the appropriate beaker
\item Prepare a beaker of DI water
\item Put the chip in the developer for exactly 45 sec
\item While in developer move the chip around gently. always hold the chip in tweezers.
\item Put the chip in water for 30sec to rince it.
\item Blow dry 10 long seconds
\item Check under microscope
\item If islands of resist remains, repeat the operation for a shorter time
sharps etched edges mean success
\item \underline{Throw the developer in the waste TMAH bottle} and rinse beakers with water and IPA
\end{enumerate}

\subsubsection{Misc}
\begin{itemize}
  \item The aim is to get sharp well defined edges, of the same size as on the mask.
  \item If features are smaller than on the mask, you can either increase exposure time or increase development time.
  \item If features are too large, either decrease exposure time or development time.
  \item The size of the undercut on LOR is defined by the amount of time it spends in developer (and bake temperature).
  It is not affected by exposure time.
\end{itemize}
\newpage

