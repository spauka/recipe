\section{Atomic Layer Deposition}
\subsubsection{Aim}
To lay down a thin, controlled layer of \ce{Al2O3}.

\subsubsection{Steps}
\begin{enumerate}
\item Open up the Savannah software on the laptop.
\item Check that the inner and outer wafer stage heaters are set to \SI{120}{\degreeCelsius}, and that the stages
are at that temperature. (note: \ce{N2} flow should be at 5 sccm) Chamber will not vent if the stage is at $<\SI{80}{\degreeCelsius}$.
\item Hit the \fbox{Vent} button to vent the deposition chamber. Wait until the software says vented.
\item Open lid using heat-proof handle and load wafers. Close chamber and hit the \fbox{Pump} button to pump down chamber.
\item Adjust the wafer stage temperature to the desired range (\SI{80}{\degreeCelsius}---\SI{200}{\degreeCelsius}).
\item Load a recipe by right-clicking in the recipe panel. Standard recipes are defined for \ce{Al2O3} for several temperatures.
\item Adjust the number of cycles to get the desired thickness. Growth rates are given in the ALD Handbook which should be near the system.
\item Run recipe by pressing \fbox{Run}
\item Once recipe is complete, temperature should be back at \SI{120}{\degreeCelsius} and \ce{N2} flow at 5 sccm. Press \fbox{Vent} to vent chamber.
\item Unload samples (remember the heat-proof handle) and restore chamber vacuum by pressing \fbox{Pump}.
\item Update logbook
\end{enumerate}

\subsubsection{Misc}
\begin{enumerate}
\item If the chamber doesn't vent, ensure that the stage heaters are on.
\item Each cycle will deposit about \SI{1}{\angstrom} of dielectric and takes about 1:20 minutes. A \SI{20}{\nano\meter} layer
will take a little over \SI{4}{\hour} to complete.
\end{enumerate}
\newpage

