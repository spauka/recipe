\section{Dektak}

\subsubsection{Aim}
Measure the thickness of remaining resist. Set up a reference to measure the etch of the mesa.

\subsubsection{Steps}
\begin{enumerate}
\item Draw the pattern of your mesa, or any geometry you want to measure the depth. Mark and number where measurement will be made.
\item Turn Dektak on 10 min before use
\item Clean the plate
\item Position the chip under the camera
\item Hit \fbox{program}, \fbox{enter} in 'scan menu'
\item Set up:
\begin{itemize}[nolistsep, noitemsep]
  \item Scan length (\SI{200}{\micro\meter})
  \item Scan speed (medium)
  \item Scan range (\SI{655}{\kilo\angstrom})
  \item Scan force (between 4 (noisy) and 9)
\end{itemize}
\item Hit \fbox{scan} to do your scan
\item Hit \fbox{ref} (\fbox{meas}) to manipulate the reference (measurement) marker.
\begin{itemize}[nolistsep, noitemsep]
  \item \noindent \fbox{lvl}: remove a slope in measurements between the ref and meas cursors
  \item \fbox{Avg HT}: average of measurements between markers
  \item \fbox{Max HT}: difference between the maximum and minimum between markers
  \item \fbox{$\Delta$ ASH}: average step height
  \item \fbox{PT}: print
\end{itemize}
\item Always draw where you did your measurements, the thickness of the resist will normally vary by \SIrange{10}{100}{\nano\meter} across the chip
\end{enumerate}

\subsubsection{Misc}
\begin{itemize}
\item After developping the resist, the height of the remaining resist should be around \SI{12}{\kilo\angstrom}
\item If measurements are 'bumpy', try increasing the force
\item If measurements look `strange' or incoherent, first re-try a measurement
\end{itemize}
\newpage

