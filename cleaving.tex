\section{Cleaving}
\subsubsection{Aim}
Cut a desired geometry from the heterostructure
\subsubsection{Needs}
\begin{itemize}[noitemsep]
\item Round shape paper filter (always manipulate on it).
\item 2 glass slides
\item diamond pen
\item heterostructure
\item clean gloves (double glove)
\end{itemize}
\subsubsection{Steps}
\begin{enumerate}
\item Write on the logbook which part of the heterostructure you are going to take (name, date, location, what for...).\\
Label your part on the logbook:\\
h(hetero)/j(junk).date(200411).initials(sb).part nb(01...)
\item Take 2 glass parts, put the heterostructure on it.
\item Scratch the chip with the diamond pen (length of scratch $<$ \SI{1}{\milli\meter}) at the wanted chip length.\\
The dimensions of the chips are:
$5 \times 2.5$ \si{\milli\meter}
\item If a wafer needs to be cut, do a scratch with the diamond pen on both ends of the slide.\\
(Note: We can use the upstairs custom cleaving device as well.)
\item Balance the GaAs chip on the glass with the scratch aligned with the edge of the glass.
\item Press GaAs with edge of finger to cleave it.
\item The cleave should be clean and along the scratch direction. Check under microscope.
\item Choose a corner as a reference according to scratches.
\item Label the position of the chip in the box, and record details in log book.
\item Put the filter paper in the contaminated bin.
\end{enumerate}

\newpage

