\section{Resist spinning}

\subsubsection{Aim}
Dispense a homogeneous layer of resist on the chip

\subsubsection{Resist Stack}
The resist stacks for various steps of processing are as follows:
\begin{description}[noitemsep, nolistsep, leftmargin=\parindent, labelindent=\parindent]
\item[Mesa Etch] \hfill
  \begin{itemize} [noitemsep, nolistsep]
    \item AZ6612
  \end{itemize}
\item[Ohmics] \hfill
  \begin{itemize} [noitemsep, nolistsep]
    \item AZ6612
    \item LOR 20B
  \end{itemize}
\item[Fine Gates] \hfill
  \begin{itemize} [noitemsep, nolistsep]
    \item PMMA A3
  \end{itemize}
\item[Optical Gates] \hfill
  \begin{itemize} [noitemsep, nolistsep]
    \item AZ6612
    \item LOR 20B
  \end{itemize}
\end{description}

\subsubsection{Needs}
\begin{itemize} [noitemsep]
\item check everything is clean
\item IPA
\item pipette
\item resist AZ6612, LOR 20B, PMMA A3
\item timer
\end{itemize}

\subsubsection{Programming}
\begin{enumerate}
  \item Hit \fbox{recipe}, then recipe number (\fbox{1}).
  \item If the recipe needs to be cleared (too many steps) hit \fbox{recipe}, \fbox{clear} and then the recipe number (\fbox{1}).
  \item Hit \fbox{step}, then the step number (\fbox{1}).
  \item Hit \fbox{speed/ramp}, then set up the rotation speed (500rpm for step 1). Hit \fbox{enter} when done.
  \item Hit \fbox{speed/ramp}, then set up the ramp slope (500rpm/sec ramp for step 1). Hit \fbox{enter} when done.
  \item Hit \fbox{terminate}, then set up the duration of the step (5.0 sec for step 1). Hit \fbox{enter} when done.
  \item Hit \fbox{step}, choose the step number and repeat as many times as required.
  \item To finish, hit \fbox{step}, then \fbox{0}.
\end{enumerate}

\subsubsection{Recipes}
\begin{description}[nolistsep, noitemsep]
\item {LOR 20B}
  \begin{enumerate}
  \item 500rpm for 5 sec, 500 rpm/sec ramp.
  \item 10000rpm for 1 sec, 10000 rpm/sec ramp.
  \item 6000 for 60 sec, 4000 rpm/sec ramp.
  \item (finish) 0 rpm for 0 sec, 2000 rpm/sec ramp.
  \item Bake for 5 min at \SI{170}{\celsius}
  \end{enumerate}
\item {AZ6612}
  \begin{enumerate}
  \item 500 rpm for 5 sec, 500 rpm/sec ramp.
  \item 10000 rpm for 20 sec, 4000 rpm/sec ramp.
  \item 4000 rpm for 20 sec, 4000 rpm/sec ramp.
  \item (finish) 0 rpm for 0 sec, 2000 rpm/sec ramp.
  \item Bake for 60 sec at \SI{95}{\celsius}
  \end{enumerate}
\item {PMMA A3}
  \begin{enumerate}
  \item 500 rpm for 5 sec, 500 rpm/sec ramp.
  \item 9000 rpm for 5 sec, 4000 rpm/sec ramp.
  \item 6000 rpm for 30 sec, 4000 rpm/sec ramp.
  \item (finish) 0 rpm for 0 sec, 2000 rpm/sec ramp.
  \item Bake for 90 sec at \SI{180}{\celsius}
  \end{enumerate}
\end{description}

\subsubsection{Steps}
\begin{enumerate}
\item Put the chip at the center of the chuck (ensure there is no wobbles).
\item Check the recipe.
\item Make sure other chips/tools are protected from projections.
\item Take a small amount of resist with a pipette. Put the pipette with resist in its bag.
\item Spin clean the chip:
\begin{enumerate}
\item Spin at 4000rpm.
\item Spray 5 seconds of Acetone
\item Spray 5 seconds with Acetone and IPA simultaneously (don't let Acetone dry)
\item Spray 5 seconds with IPA
\item Blow dry while spinning
\end{enumerate}
\item Place 2-3 drops of resist with the pipette on the chip during the 500 rpm spin
\item Always put the pipette back in its bag, wait for the spinning to stop
\item Bake the chip
\item Put the chip back in its box
\item Clean the spinner (!no liquid in vacuum hole!): use wipes with acetone to clean the chuck, protections, bench. Note that
      if LOR 20B was used, clean with a wipe soaked in NMP prior to the acetone, as LOR 20B will precipitate and crystalize on the spinner.
\end{enumerate}

\subsubsection{Misc}
\begin{itemize}
\item Don't put liquid in the spinner vacuum hole when cleaning.
\item Resist AZ6612, 12 means \SI{1.2}{\micro\meter}. (Measured depth closer to \SI{800}{\nano\meter})
\item If you will be spinning LOR 20B turn on the hot plate before you start. It takes about 10 minutes to warm up.
\item If LOR 20B has been spun, remember to clean using NMP prior to using acetone.
\item Note that the vacuum only turns on while the spinner is running. Don't spray before or after the spinner has stopped.
\end{itemize}

\newpage

