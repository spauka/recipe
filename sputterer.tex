\section{HHV Sputterer}

\textbf{Aim}:
Deposit a thin layer of metal using the HHV Sputterer

\subsubsection{Needs}
\begin{itemize}[noitemsep]
\item chip with developed resist
\item required source
\end{itemize}

\subsubsection{Procedure}
\begin{enumerate}
\item Press \boxed{Seal} to close the cryopump gate valve.
\item Press \boxed{Vent} to vent the chamber to atmosphere. This process takes about 3 minutes to complete.
\item Once the chamber is open check whether you need to replace sources. You can check which source is in position \#5 in HHV logbook. Sources are as follows (front-left clockwise):
\begin{enumerate}[label=Source~\#\arabic*:]
  \item Gold (Au)
  \item Chromium (Cr)
  \item Aluminium (Al)
  \item Titanium (Ti)
  \item User Changable
\end{enumerate}
\end{enumerate}

\noindent \textbf{Misc.}:
\begin{itemize}
\item If liftoff was unsuccessful, you can attempt the above steps again. However once dry, metal becomes
much more difficult to lift off.
\item Spraying vigorously with acetone can help liftoff along.
\item If gates are broken, check the edges under SEM. Bright white edges suggests the metal was ripped
off during liftoff and is an indicator of insufficient resist depth.
\end{itemize}
\newpage

