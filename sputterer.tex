\section{HHV Sputterer}

\textbf{Aim}:
Deposit a thin layer of metal using the HHV Sputterer

\subsubsection{Needs}
\begin{itemize}[noitemsep]
\item chip with developed resist
\item required source
\end{itemize}

\subsubsection{Procedure}
\begin{enumerate}
\item Press \boxed{Seal} to close the cryopump gate valve.
\item Press \boxed{Vent} to vent the chamber to atmosphere. This process takes about 3 minutes to complete.
\item Once the chamber is open check whether you need to replace sources. You can check which source is in position \#1/\#5 in the HHV logbook. Other sources are as follows (front-left clockwise):
\begin{enumerate}[label=Source~\#\arabic*:]
  \item User Changeable
  \item Chromium (Cr)
  \item Aluminium (Al)
  \item Titanium (Ti)
  \item User Changeable
\end{enumerate}
\item To change the source:
\begin{enumerate} [noitemsep,nolistsep]
  \item Open the source shutter by setting a long source opening time and clicking on \boxed{Shutter [X]} corresponding to the source you wish to change.
  \item Remove the chimney around the source, an allen key of the correct size should be placed inside the sputterer.
  \item Using the red screwdriver, lever the source off the stack. This requires a bit of force as it is held on by a strong magnet.
  \item Clean the back of the source and the top of the stack using a clean room wipe moistened with IPA. Ensure all thermal grease is removed.
  \item Place the removed source into its foil cover and place it into the sputtering chamber.
  \item With the tip of your glove, spread some thermal paste around the raised edge of the top of the stack.
  \item Clean the back of the new source you wish to install with IPA. Place the new source on top of the stack. Be aware of the strong magnet in the stack as you do so.
  Adjust the stack such that the trimmed rules is touching both the center of the source and the 6 inch mark of the chip holder plate.
  \item Place the chimney back on and tighten the screws, ensuring that the source is not shorting against any of the edges of the chimney.
  \item Close the shutter by pressing the \boxed{Shutter [X]} button again.
\end{enumerate}
\item Ensure that the shutters for each of the sources that you wish to use opens.
\item Mount your chip on the chip plate. It can be removed and reinserted by rotating the plate. The allen key for this process is the same as for the E-Beam evaporator
      and is normally kept on the table for that tool.
\item Close the chamber and press \boxed{Seal} and \boxed{HV Pumping} to begin the pumpdown process. Wait until the pressure is below \SI{3e-6}{\torr}. This normally takes between 20 and 40 minutes.
      Edit the recipe while you wait.
\item Configure the recipe for each metal you want to use:
\begin{enumerate} [noitemsep,nolistsep] % TODO: Confirm all these items are correct. Filled in from memory....
  \item Hit the \boxed{Recipe} menu item and then hit \boxed{Edit Recipe}.
  \item Select the recipe you wish to edit and hit \boxed{Edit Recipe}.
  \item A list of steps is presented. Select the step you wish to edit and hit \boxed{Edit Step}. For single metals, the last step is normally the deposition.
  \item In the step window, edit the following items:
  \begin{itemize} [noitemsep, nolistsep]
    \item Check the source shutter is correct.
    \item Edit the source shutter opening time.
    \item Edit the conditioning time and power (Note, conditioning time is the final item).
    \item Edit the deposition power.
  \end{itemize}
  \item Hit save before returning. This gives no feedback so it's worth double checking that the parameters are saved. Close the window.
  \item Hit save on the recipe step. This should give confirmation that the recipe is saved.
\end{enumerate}
\item Once the vacuum stage is complete, open the \boxed{Mode} menu and put the tool into \boxed{Auto} mode.
\item Hit \boxed{Recipe}$\rightarrow$\boxed{Select Recipe} and select the recipe you would like to run.
\item In the bottom left, hit \boxed{Start} to begin the processing. Ensure that the argon begins to flow at the correct rate (15 sccm).
      Once deposition begins, ensure that you can see a ring of plasma around the source being sputtered. Open the PSU monitor and record
      relevant parameters in the sputterer logbook.
\item Once the sputtering is complete, wait for the source to cool down. A rough rule of thumb is to wait half of the sputtering time for this
      to occur.
\item Hit \boxed{Seal} and \boxed{Vent} to vent the sputterer. Remove samples and return the chamber the vacuum.
\end{enumerate}

\noindent \textbf{Misc.}:
\begin{itemize}
\item The time you should wait for the sources to cool depends on both the type of source (metal or oxide) and how reactive it is.
      Less reactive metals will need less time, while oxides should be left for longer as they are prone to cracking.
\item Ensure that you do not touch the surface of the sputtering sources even with gloved hands.
\item You only need to apply a small amount of thermal grease. Excess grease will prevent proper heat transfer.
\end{itemize}
\newpage

